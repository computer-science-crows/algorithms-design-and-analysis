%%%%%%%%%%%%%%%%%%%%%%%%%%%%%%%%%%%%%%%%%
% Journal Article
% LaTeX Template
% Version 1.4 (15/5/16)
%
% This template has been downloaded from:
% http://www.LaTeXTemplates.com
%
% Original author:
% Frits Wenneker (http://www.howtotex.com) with extensive modifications by
% Vel (vel@LaTeXTemplates.com)
%
% License:
% CC BY-NC-SA 3.0 (http://creativecommons.org/licenses/by-nc-sa/3.0/)
%
%%%%%%%%%%%%%%%%%%%%%%%%%%%%%%%%%%%%%%%%%

%----------------------------------------------------------------------------------------
%	PACKAGES AND OTHER DOCUMENT CONFIGURATIONS
%----------------------------------------------------------------------------------------

\documentclass[10pt]{article} % Single column

%\documentclass[twoside,twocolumn]{article} % Two column

\usepackage{blindtext} % Package to generate dummy text throughout this template 

\usepackage[sc]{mathpazo} % Use the Palatino font
\usepackage[T1]{fontenc} % Use 8-bit encoding that has 256 glyphs
\linespread{1.05} % Line spacing - Palatino needs more space between lines
\usepackage{microtype} % Slightly tweak font spacing for aesthetics

\usepackage[spanish]{babel} % Language hyphenation and typographical rules
\selectlanguage{spanish} 
\usepackage[hmarginratio=1:1,top=32mm,columnsep=20pt]{geometry} % Document margins
\usepackage[hang, small,labelfont=bf,up,textfont=it,up]{caption} % Custom captions under/above floats in tables or figures
\usepackage{booktabs} % Horizontal rules in tables

\usepackage{lettrine} % The lettrine is the first enlarged letter at the beginning of the text

\usepackage{enumitem} % Customized lists
\setlist[itemize]{noitemsep} % Make itemize lists more compact

\usepackage{abstract} % Allows abstract customization
\renewcommand{\abstractnamefont}{\normalfont\bfseries} % Set the "Abstract" text to bold
\renewcommand{\abstracttextfont}{\normalfont\small\itshape} % Set the abstract itself to small italic text

\usepackage{titlesec} % Allows customization of titles
\renewcommand\thesection{\Roman{section}} % Roman numerals for the sections
\renewcommand\thesubsection{\roman{subsection}} % roman numerals for subsections
\titleformat{\section}[block]{\large\scshape\centering}{\thesection.}{1em}{} % Change the look of the section titles
\titleformat{\subsection}[block]{\large}{\thesubsection.}{1em}{} % Change the look of the section titles

\usepackage{fancyhdr} % Headers and footers
\pagestyle{fancy} % All pages have headers and footers
\fancyhead{} % Blank out the default header
\fancyfoot{} % Blank out the default footer
\fancyhead[C]{Dise\~no y An\'alisis de Algoritmos. \textbf{Proyecto \# 1: La Pelota}} % Custom header text
\fancyfoot[RO,LE]{\thepage} % Custom footer text

\usepackage{titling} % Customizing the title section

\usepackage{hyperref} % For hyperlinks in the PDF

\usepackage{graphicx} % For images

\usepackage{pifont} % bullets

\usepackage{amsmath}

%\usepackage{algorithm}
%\usepackage{algorithmic}
\usepackage{algorithm2e}

% Keywords command
\providecommand{\keywords}[1]
{
	\small	
	\vspace{0.5em}
	\noindent \textbf{\textit{Palabras clave --- }} #1
}


%----------------------------------------------------------------------------------------
%	TITLE SECTION
%----------------------------------------------------------------------------------------

\setlength{\droptitle}{-4\baselineskip} % Move the title up

\pretitle{\begin{center}\Huge\bfseries} % Article title formatting
	\posttitle{\end{center}} % Article title closing formatting
\title{\normalsize{Dise\~no y An\'alisis de Algoritmos }\\
	\Huge\bfseries Proyecto \# 1: La Pelota \\
} % Article title
\author{% 
	\includegraphics[width=15em]{logo.png}\\
	Laura Victoria Riera P\'erez\\
	Mari\'e del Valle Reyes \vspace{1em} \\
	\small Cuarto a\~no. Ciencias de la Computaci\'on. \\ % institution
	\small Facultad de Matem\'atica y Computaci\'on, Universidad de La Habana, Cuba \\ % institution
}
\date{\footnotesize \today } % Leave empty to omit a date


% Abstract configurations
\renewenvironment{abstract}
{\small
	\begin{center}
		\bfseries \abstractname\vspace{-.5em}\vspace{0pt}
	\end{center}
	\list{}{
		\setlength{\leftmargin}{1.5cm}%
		\setlength{\rightmargin}{\leftmargin}%
	}%
	\item\relax}
{\endlist}

\usepackage{amsthm}
\usepackage{amssymb}
\usepackage{todonotes} % \TODO
\usepackage{listings} % Code listings
\usepackage{xcolor}

\definecolor{backcolour}{rgb}{0.95,0.95,0.92}

\newcommand{\csl}[1]{\colorbox{backcolour}{\texttt{#1}}}

\newcommand{\imgcaption}[2]{\tiny \textbf{Figura #1.} #2.}

\newcommand{\mgc}[2][]{\colorbox{backcolour}{\texttt{\_\_#2\_\_#1}}}

\newcommand{\mgccapt}[1]{\texttt{\_\_#1\_\_}}

\newtheorem{thm}{Teorema}
\newtheorem{mydef}{Definici\'on}%[section]
\newtheorem{lem}{Lema}

\renewcommand{\qedsymbol}{\rule{0.7em}{0.7em}}



% Hyperlinks configurations
\hypersetup{
	colorlinks=true,
	linkcolor=black,
	filecolor=magenta,      
	urlcolor=cyan,
	pdftitle={Overleaf Example},
	pdfpagemode=FullScreen,
}

%----------------------------------------------------------------------------------------

\begin{document}
	% Print the title
	\maketitle
	
	%----------------------------------------------------------------------------------------
	%	ARTICLE CONTENTS
	%----------------------------------------------------------------------------------------
	
	\section{Repositorio del proyecto}
	
	\begin{center}
		\href{https://github.com/computer-science-crows/algorithms-design-and-analysis}{https://github.com/computer-science-crows/algorithms-design-and-analysis}
	\end{center}

	\section{Definici\'on inicial del problema} 
	
	Para un campeonato de pelota, el manager debe elegir de un conjunto de $ n $ personas, a su equipo de $ p $ jugadores, y a $ k $ espectadores especiales para que suban la moral del equipo.
	De cada persona $ i $, el manager conoce el valor que aporta a la moral del equipo $ a_i $ y el valor que aporta siendo situado en la posici\'on $ j $, $ s_{i,j} $. 
	Determine una alineación entre jugadores en el campo y espectadores de forma que el equipo tenga la mayor cantidad de valor acumulado posible.
	
	\section{Definici\'on en t\'erminos matem\'atico - computacionales}
	
	\subsection{Preliminares}
	
	\begin{mydef}
		Grafo bipartito
	\end{mydef}
	
	\begin{mydef}
		Grafo bipartito completo
	\end{mydef}
	
	\begin{mydef}
		Para un grafo no dirigido $G = (V,E)$, un emparejamiento es un subconjunto de aristas $M \in E$ tal que cada v\'ertice en $V$ tiene al menos una a arista incidente en $M$.
	\end{mydef}

	\begin{mydef}
		Otras definiciones
	\end{mydef}

	\begin{mydef}
		Camino M-alternativo
	\end{mydef}
	
	\begin{mydef}
		Camino M-aumentativo
	\end{mydef}
	
	\subsection{Problema de asignaci\'on}
	
	\section{Posibles soluciones investigadas}
	
	\section{L\'inea de pensamiento}
	\subsection{Backtrack}
	\subsection{Greedy}
	\subsection{Simplex}
	\subsection{Max flow min cut para mayor emparejamiento}
	No maximiza segun costos de aristas
	
	Se decidi\'o implementar el Hungarian \todo{referencia al introduction} por ser la solucion existente que resuelve lo pensado. 
	
	
	\section{Hungarian algorithm}
	Dado un grafo bipartito completo ponderado $G = (V,E)$, donde $V = L \cup R$. Se asume que los v\'ertices de los conjuntos $L$ y $R$ contienen $n$ v\'ertice cada uno, por tanto el grafo contiene $n^2$ aristas. Para $l \in L$ y $r \in R$, se denota el peso de la arista $(l,r)$ como $w(l,r)$, lo cual representa ganancia de emparejar el v\'ertice $l$ con el v\'ertice $r$.
	
	El objetivo es encontrar el emparejamiento perfecto $M*$ cuyas aristas tengan el peso m\'aximo total de todos los emparejamientos perfectos posibles. 
	
	Sea $w(M) = \sum_{(l,r) \in M} w(l,r)$ el peso total de las aristas en el emparejamiento $M$, se quiere encontrar el emparejamiento perfecto $M^*$ tal que,
	\[w(M*)=\text{max}\{w(M):M \text{ es un emparejamiento perfecto}\}\].
	
	A encontrar un emparejamiento perfecto de peso m\'aximo se le llama \textbf{problema de asignaci\'on}. Una soluci\'on del problema de asignaci\'on es un emparejamiento perfecto que maximice el costo total.
	
	Aunque se pueden enumerar los $n!$ emparejamientos perfectos pra resolver este problema, existe un algoritmo llamado \textbf{algoritmo H\'ungaro} que lo resuelve m\'as r\'apido. En vez de trabajar con un grafo bipartito completo $G$, el algoritmo H\'ungaro trabaja con un subgrafo de $G$ llamado \textbf{subgrafo de igualdad}. El subgrafo de igualdad cambia en el tiempo y tiene la propiedad que cualquier emparejamiento perfecto en el subgrafo de igualdad es tambi\'en una soluci\'on \'optima del problema de asignaci\'on.
	
	El subgrafo de igualdad depende de asignar un atributo $h$ a cada v\'ertice. El atributo $h$ se llama \textbf{etiqueta} del v\'ertice. 
	
	Se dice que $h$ es un \textbf{etiquetado de vértice factible} de $G$ si $l.h + r.h \geq w(l,r)$ para todo $l \in L$ y $r \in R$.
	
	Un etiquetado de v\'ertice factible siempre existe, como el \textbf{etiquetado de v\'ertice por defecto} dado por
	\begin{align}
		\label{eq:defecto}
		l.h &= \text{max} \{w(l,r):r \in R\} &\text{para todo } l \in R,\\
		r.h &= 0 &\text{para todo } r \in R 
	\end{align}

	Dado un eqiquetado de v\'ertice factible $h$, el \textbf{subgrafo de igualdad} $G_h = (V, E_h)$ de $G$ consiste de los mismos v\'ertice de $G$ y el subconjunto de aristas $E_h = \{(l,r) \in E: l.h + r.h = w(l,r)\}$.
	
	\begin{thm}
		
		Sea $G=(V,E)$, donde $V = L \cup R$, un grafo bipartito completo donde cada arista $(l,r) \in E$ tiene peso $w(l,r)$. Sea $h$ un etiquetado de v\'ertice factible de $G$ y $G_h$ el subgrafo de igualdad de $G$. Si $G_h$ contiene un emparejamiento perfecto $M^*$, entonce $M^*$ es una soluci\'on \'optima del problema de asignaci\'on $G$.
		
	\end{thm}

	\begin{proof}
		Si $G_h$ tiene un emparejamiento perfecto $M^*$, entonces debido a que $G_h$ y $G$ tienen el mismo conjunto de v\'ertices, $M^*$ es tambi\'en un emparejamiento perfecto en $G$. Debido a que cada arista de $M^*$ pertenece a $G_h$ y cada v\'ertice tiene exactamente una arista incidente del emparejamiento perfecto, entonces se tiene
		
		\begin{align}
			w(M*) &= \sum_{(l,r) \in M*} w(l,r)\\
			&= \sum_{(l,r) \in M^*}(l.h + r.h) &\text{(porque todas las aristas de $M^*$ pertenecen a $G_h$)}\\
			&= \sum_{l \in L}l.h + \sum_{r \in R} r.h &\text{(porque $M^{*}$ es un emparejamiento perfecto)}\\
		\end{align} 
		
		Sea $M$ un emparejamiento perfecto cualquiera de $G$, se tiene
		
		\begin{align}
			w(M) &= \sum_{(l,r) \in M} w(l,r)\\
			&\leq \sum_{(l,r) \in M} (l.h + r.h) &\text{(porque $h$ es un etiquetado de v\'ertice factible)}\\
			&= \sum_{l \in L} l.h + \sum_{r \in R} r.h &\text{(porque $M$ es un emparejamiento perfecto)}
		\end{align}
		Entonces se tiene
		\begin{equation}
			w(M) \leq \sum_{l \in L} l.h + \sum_{r \in R} r.h = w(M^*),
		\end{equation}
		por tanto $M^*$ es un emparejamiento perfecto de m\'aximo costo en $G$.
	\end{proof}

	El objetivo ahora ser\'ia encontrar un emparejamiento perfecto en un subgrafo de igualdad. 

Si se elige cualquier conjunto de etiquetas de v\'ertices que defina un subgrafo de igualdad, entonces un emparejamiento de cardinalidad m\'axima en este subgrafo tiene un valor total de a lo sumo la suma de las etiquetas de los v\'ertices. Si el conjunto de etiquetas de v\'ertices es el 'correcto', entonces tendr\'a un valor total igual a $w(M^*)$, y un emparejamiento de cardinalidad m\'axima es el subgrafo de igualdad es tambi\'en un emparejamiento perfecto de m\'aximo peso.

\todo[inline]{Explicaci\'on del algoritmo H\'ungaro}
El algoritmo H\'ungaro repetidamente modifica el emparejamiento y las etiquetas de v\'ertices en orden de alcanzar su objetivo.

El algoritmo H\'ungaro empieza con un etiquetado de v\'ertice factible $h$ y cualquier emparejamiento $M$ en el subgrafo de igualdad $G_h$. Repetidamente encuentra un camino $M$-aumentativo $P$ en $G_h$ y utilizando el Lema 25.1 \todo{Escribir lema}, actualiza el emparejamiento para que sea la diferencia sim\'etrica de $M$ y $P$, incrementando as\'i el tama\~no del emparejamiento. Mientras haya alg\'un subgrafo de igualdad que contenga un camino $M$- aumentativo, el tama\~no del emparejamiento puede incrementar, hasta que un emparejamiento perfecto se logre.

\todo[inline]{Caracteristicas del algoritmo}
\begin{itemize}
	\item El etiquetado de v\'ertices inicial del algoritmo es el etiquetado de v\'ertice por defecto de \ref{eq:defecto}.
	\item El emparejamiento de $G_h$ inicial puede ser cualquier emparejamiento. En la resoluci\'on de este problema se utiliz\'o un algoritmo de emparejamiento maximal greedy.
	\item  Para encontrar un camino $M$ - aumentativo en $G_h$, se utiliza una variante de BFS.
	\item Si la b\'usqueda de un camino $M$-aumentativo falla, se debe actualizar el etiquetado de v\'ertice factible para adicionar a $G_h$ al menos una arista nueva.
\end{itemize}		

	\subsection{Encontrar un camino $M$-aumentativo en $G_h$}
	
	Un camino $M$-aumentativo es aquel que empieza en un v\'ertice no saturado de $L$, termina en un v\'ertice no saturado de $R$, tomando aristas no saturadas de $L$ a $R$ y aristas saturadas de $R$ a $L$.
	
	El algoritmo H\'ungaro busca un camino $M$-aumentativo desde cualquier v\'ertice no saturado de $L$ hacia cualquier v\'ertice no saturado de $R$. 
	
	Como m\'etodo de b\'usqueda exhaustiva en un grafo se utiliza el BFS. El algoritmo empieza la b\'usqueda desde todos los v\'ertices no saturados de $L$, los cuales al inicio se insertan en la cola $Q$. La condici\'on de parada es que descubra alg\'un v\'ertice no saturado de $R$. El resultado del algoritmo es un bosque primero a lo ancho $F = (V_f, E_f)$, donde cada v\'ertice no saturado de $L$ es ra\'iz de alg\'un \'arbol de $F$.
	
	\subsection{Cuando falla la b\'usqueda del camino M-aumentativo}
	
	Una vez actualizado el emparejamiento $M$ con el camino $M$-aumentativo, el algoritmo H\'ungaro actualiza el subgrafo de igualdad $G_h$ de acuerdo al nuevo emparejamineto y luego empieza una nueva b\'usqueda desde los v\'ertices no saturados de $L$.
	
	Existen casos en los que la cola $Q$ se vac\'ia sin que se halla llegado a encontrar un v\'ertice no saturado de $R$ que conforme un camino $M$-aumentativo. Cuando esto ocurre, el algoritmo H\'ungaro actualiza el etiquetado de v\'ertices factible $h$ de acuerdo al siguiente lemma.
	
	\begin{lem}
		
		Sea $h$ un etiquetado de v\'ertice factible en el grafo bipartito completo G con el grafo de igualdad $G_h$, y se $M$ un emparejamiento para $G_h$ y $F$ el bosque constru\'ido a partir de una B\'usqueda Primero a lo Ancho (en ingl\'es, BFS) sobre el subgrafo de igualdad $G_h$. Entonces, la etiqueta $h'$,
		\begin{equation}
			v.h' =  \left\{
			\begin{array}{ll}
				v.h-\delta & \text{si $v \in F_l$}, \\
				v.h+\delta & \text{si $v \in F_r$}, \\
				v.h & e.o.c \\
			\end{array} 
			\right.
		\end{equation}				
		donde 
		\begin{equation}
			\label{eqn:delta}
		\delta = min\{l.h + r.l - w(l,r): l\in F_l, r \in F_r\}
		\end{equation}
		con $F_l = L \cap V_f$ y $F_r = R \cap V_f$ son v\'ertices del bosque $F$ que pertenecen a $L$ y a $R$, respectivamente, es una etiqueta de v\'ertice factible para $G$ con las siguientes propiedades:
		\begin{enumerate}
			\item Si (u,v) es una arista de bosque F para $G_h$, entonces (u,v) $\in E_{h'}$.
			\item Si (l,r) pertenece al emparejamiento M para $G_h$, entonces (l,r) $\in E_{h'}$.
			\item Existen v\'ertices $l \in F_l$ y $r \in R - F_r$ tales que (l,r) $\notin E_h$, pero (l,r) $\in E_{h'}$.  
		\end{enumerate} 
	\end{lem}
	\begin{proof}
		Primero se demuestra que $h'$ es un etiquetado de v\'ertice factible para $G$. Debido a que $h$ es un etiquetado de v\'ertice factible, se tiene $l.h + r.h \geq w(l,r)$ para todo $l \in L$ y $r \in R$. Para que $h'$ no sea un etiquetado de v\'ertice factible, entonces se necesitar\'ia que $l.h' + r.h' < l.h + r.h$ para alg\'un $l \in L$ y $r \in R$. La \'unica forma en que esto pudiera ocurrir ser\'ia para alg\'un $l \in F_l$ y $r \in R-F_r$. En esta instancia, la cantidad de decrecimiento es igual a $\delta$, entonces $l.h' + r.h' = l.h - \delta + r.h$. Por ecuaci\'on \ref{eqn:delta}, se tiene que $l.h - \delta + r.h \geq w(l,r)$ para cualquier $l \in F_l$ y $r \in R - F_r$, por tanto $l.h' + r.h' \geq w(l,r)$. Para cualquier otra arista, se tiene $l.h' + r.h' \geq l.h + r.h \geq w(l,r)$. Por tanto, $h'$ es un etiquetado de v\'ertice factible.
		
		Ahora se mostrar\'a la veracidad de las propiedades:
		
		\begin{enumerate}
			\item Si $l \in F_l$ y $r \in F_r$, entonces se tiene $l.h' + r.h' = l.h + r,h$ debido a que $\delta$ se adiciona a la etiqueta de $l$ y se substrae de la etiqueta de $r$. entonces, si una arista pertenece a $F$ para el grafo $G_h$, tambi\'en pertenece a $G_{h'}$.
			\item Se afirma que para el momentp en que el algoritmo H\'ungaro computa el nuevo etiquetado de v\'ertice factible $h'$, para toda arista $(l,r) \in M$, se tiene que $l \in F_l$ si y solo s\'i $r \in F_r$. 
			
			Para demostrar por qu\'e, se considera el v\'ertice saturado $r$ y la arista $(l,r) \in M$. 
			
			Primero se supone que $r \in F_r$, entonces la b\'usqueda encuentra $r$ y lo pone en la cola. Cuando $r$ se remueve de la cola, $l$ es descubierto, entonces $l \in F_l$. 
			
			Luego se supone que $r \notin F_r$, por tanto $r$ no se ha descubierto. Se demostrar\'a que $l \notin F_l$. La \'unica arista en $G_h$ que entra $l$ es $(r,l)$, y dado que $r$ no se ha descubierto, la b\'usqueda no ha tomado esta arista; si $l \in F_l$, no es por la arista ($r,l$). La \'unica otra forma que un v\'ertice en $L$ puede estar en $F_l$ es si es ra\'iz de la b\'usqueda, pero solo v\'ertices no saturados de $L$ son ra\'ices y $l$ est\'a saturado. Por tanto, $l \notin F_l$ y la afirmaci\'on se cumple.
			
			Se conoce que para $l \in F_l$ y $r \in F_r$ se cumple $l.h' + r.h' = l.h + r.h$. En caso contrario, cuando $l \in L-F_l$ y $r \in R-F_r$, se tiene que $l.h' = l.h$ y $r.h'=r.h$, entonces $l.h' + r.h' = l.h + r.h$. Por tanto, si la arista $(l,r)$ est\'a en el emparejamiento $M$ para el grafo $G_h$, entonces $(l,r) \in E_{h'}$.
			\item Sea $(l,r)$ una arista que no pertenece a $E_h$, tal que $l \in F_l$, $r \in R-F_r$ y $\delta = l.h + r.h - w(l,r)$. Entonces, por definici\'on de $\delta$, existe al menos una de esas arista. Luego, se tiene 
			
			\begin{align*}
				l.h' + r.h' &= l.h - \delta + r.h\\
				&= l.h - (l.h + r.h -w(l,r)) + r.h\\
				&= w(l,r)			
			\end{align*}
		y por tanto $(l,r) \in E_{h'}$. Dado que $(l,r) \notin E_h$, no est\'a en el emparejamiento $M$, entonces en $E_{h'}$ debe estar dirigida de $L$ a $R$\todo{arreglar esto porque nuestro grafo no es dirigido}. Por tanto, $(l,r) \in E_{h'}$.
			
			
		\end{enumerate}
		
	\end{proof}

	Es posible que una arista pertenezca a $E_h$ pero no a $E_{h'}$.\todo{demostrar esto ex 25.3-3}
	
	\subsection{Correctitud}
	
	\subsection{Complejidad Temporal}
	
	La complejidad temporal del algoritmo H\'ungaro implementado es $O(n^4)$, donde $|V|=2n$ y $|E|=n^2$ en el grafo original $G$.
	
	
	
	
	
	
	\subsection{Complejidad Espacial}
	
	\section{Generador de casos de prueba}
	
	\section{Tester}
	
	
	\section{Comparaci\'on de soluciones implementadas}
	
	\begin{thebibliography}
		a
		\bibitem{introduction} Cormen, Thomas H. y otros. \emph{Introduction to Algorithms}. 
		The MIT Press.
		4ta Edici\'on.		
		Cambridge, Massachusetts.
		2022.
	\end{thebibliography}
\end{document}


